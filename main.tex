% Pengaturan ukuran teks dan bentuk halaman dua sisi
\documentclass[12pt]{report}

% Pengaturan ukuran halaman dan margin
\usepackage[a4paper,top=30mm,left=30mm,right=20mm,bottom=25mm]{geometry}

% Pengaturan ukuran spasi
\usepackage[singlespacing]{setspace}

% Judul dokumen
\title{Proposal Tugas Akhir ITS}
\author{Musk, Elon Reeve}

% Pengaturan format bahasa
\usepackage[indonesian]{babel}
  \addto{\captionsindonesian}{\renewcommand{\bibname}{Daftar Pustaka}}

% Pengaturan detail pada file PDF
\usepackage[pdfauthor={\@author},bookmarksnumbered,pdfborder={0 0 0}]{hyperref}

% Pengaturan jenis karakter
\usepackage[utf8]{inputenc}

% Pengaturan ukuran indentasi
\setlength{\parindent}{2em}

% Package lainnya
\usepackage{changepage}
\usepackage{etoolbox} % Mengubah fungsi default
\usepackage{apacite} % APA-style citation
\usepackage{enumitem} % Pembuatan list
\usepackage{lipsum} % Pembuatan template kalimat
\usepackage{graphicx} % Input gambar
\usepackage{longtable} % Pembuatan tabel
\usepackage[table,xcdraw]{xcolor} % Pewarnaan tabel
\usepackage[numbers]{natbib} % Pengaturan kutipan artikel
\usepackage{eso-pic} % Untuk menggunakan background image di halaman
\usepackage{txfonts} % Font times
\usepackage{changepage} % Pembuatan teks kolom
\usepackage{multicol} % Pembuatan kolom ganda
\usepackage{multirow} % Pembuatan baris ganda
\usepackage{tabularx} % Untuk mengatur kolom, seperti grid pada CSS
\usepackage{wrapfig}

% Definisi untuk "Hati ini sengaja dikosongkan"
\patchcmd{\cleardoublepage}{\hbox{}}{
  \thispagestyle{empty}
  \vspace*{\fill}
  \begin{center}\textit{[Halaman ini sengaja dikosongkan]}\end{center}
  \vfill}{}{}

  % Pengaturan penomoran halaman
\usepackage{fancyhdr}
\fancyhf{}
\renewcommand{\headrulewidth}{0pt}
\pagestyle{fancy}
\fancyfoot[C,CO]{\thepage}
\patchcmd{\chapter}{plain}{fancy}{}{}
\patchcmd{\chapter}{empty}{plain}{}{}

% Pengaturan format judul bab
\usepackage{titlesec}
\renewcommand{\thesection}{\arabic{section}}
\titleformat{\chapter}[display]{\bfseries\Large}{BAB \centering\Roman{chapter}}{0ex}{\vspace{0ex}\centering}
\titleformat*{\section}{\large\bfseries}
\titleformat*{\subsection}{\normalsize\bfseries}
\titlespacing{\section}{0ex}{3ex}{1.5ex}
\titlespacing{\subsection}{0ex}{3ex}{1.5ex}
\titlespacing{\subsubsection}{0ex}{0.5ex}{0ex}

% Isi keseluruhan dokumen
\begin{document}
  % Nomor halaman pembuka dimulai dari sini
  \pagenumbering{roman}

  % Atur ulang penomoran halaman
  \setcounter{page}{1}

  % Sampul Bahasa Indonesia
  \newcommand\covercontents{sampul/konten-id.tex}
  \input{sampul/sampul-luar.tex}

  % Sampul Bahasa Inggris
  \renewcommand\covercontents{sampul/konten-en.tex}
  \input{sampul/sampul-luar.tex}

  % Lembar pengesahan
  \chapter*{LEMBAR PENGESAHAN}

% Menyembunyikan nomor halaman
\thispagestyle{empty}

\begin{center}
  % Ubah kalimat berikut dengan judul tugas akhir
  \textbf{KALKULASI ENERGI PADA ROKET LUAR ANGKASA BERBASIS \emph{ANTI-GRAVITASI}}
\end{center}

\begingroup
% Pemilihan font ukuran small
\small

\begin{center}
  % Ubah kalimat berikut dengan pernyataan untuk lembar pengesahan
  \textbf{PROPOSAL TUGAS AKHIR} \\
  Diajukan untuk memenuhi salah satu syarat memperoleh gelar
  Sarjana Teknik pada
  Program Studi S-1 Teknik Dirgantara \\
  Departemen Teknik Dirgantara \\
  Fakultas Teknik Dirgantara \\
  Institut Teknologi Sepuluh Nopember
\end{center}

\begin{center}
  % Ubah kalimat berikut dengan nama dan NRP mahasiswa
  Oleh: \textbf{Elon Reeve Musk} \\
  NRP. 0123 20 4000 0001
\end{center}

\begin{center}
  Disetujui Oleh:
\end{center}

\vspace{10ex}

\begingroup
% Menghilangkan padding
\setlength{\tabcolsep}{0pt}

\noindent
\begin{tabularx}{\textwidth}{X c}
  % Ubah kalimat-kalimat berikut dengan nama dan NIP dosen pembimbing pertama
  Nikola Tesla, S.T., M.T.      &                 \\
  NIP: 18560710 194301 1 001    & (Pembimbing)    \\
                                &                 \\
                                &                 \\
                                &                 \\
  % Ubah kalimat-kalimat berikut dengan nama dan NIP dosen pembimbing kedua
  Wernher von Braun, S.T., M.T. &                 \\
  NIP: 19230323 197706 1 001    & (Ko-Pembimbing) \\
\end{tabularx}
\endgroup

\vspace{\fill}

\begin{center}
  Mengetahui,\\
  % Ubah kalimat berikut dengan nama departemen
  Kepala Departemen Teknik Dirgantara FTD-ITS\\
  \vspace{10ex}
  % Ubah kalimat berikut dengan jabatan kepala departemen
  \underline{Nikola Tesla, S.T., M.T. }\\
  NIP 18560710 194301 1 001\\
  \vspace{10ex}
  % Ubah text dibawah menjadi tempat dan tanggal
  \textbf{SURABAYA} \\
  \textbf{Mei, 2077}
\end{center}
\endgroup

  \newpage

  % Lembar pengesahan
  \begin{center}
	\large
  \textbf{APPROVAL SHEET}
\end{center}

% Menyembunyikan nomor halaman
\thispagestyle{empty}

\begin{center}
  % Ubah kalimat berikut dengan judul tugas akhir
  \textbf{\emph{ANTI-GRAVITY} BASED ENERGY CALCULATION ON OUTER SPACE ROCKETS}
\end{center}

\begingroup
  % Pemilihan font ukuran small
  \small

  \begin{center}
    % Ubah kalimat berikut dengan pernyataan untuk lembar pengesahan
    \textbf{FINAL PROJECT PROPOSAL} \\
    Submitted to fulfill one of the requirements for obtaining a degree
    Bachelor of Engineering at 
    Undergraduate Study Program of Aerospace Engineering \\
    Department of Aerospace Engineering \\
    Faculty of Aerospace Technology \\
    Sepuluh Nopember Institute of Technology
  \end{center}

  \begin{center}
    % Ubah kalimat berikut dengan nama dan NRP mahasiswa
    By: \textbf{Elon Reeve Musk} \\
    NRP. 0123 20 4000 0001
  \end{center}

  \begin{center}
    Approved by Final Project Proposal Examiner Team:
  \end{center}

  \begingroup
    % Menghilangkan padding
    \setlength{\tabcolsep}{0pt}

    \noindent
    \begin{tabularx}{\textwidth}{X c}
      % Ubah kalimat-kalimat berikut dengan nama dan NIP dosen pembimbing pertama
      Nikola Tesla, S.T., M.T.          & (Advisor) \\
      NIP: 18560710 194301 1 001        & \\
      &  \\
      &  \\
      % Ubah kalimat-kalimat berikut dengan nama dan NIP dosen pembimbing kedua
      Wernher von Braun, S.T., M.T.     & (Co-Advisor) \\
      NIP: 19230323 197706 1 001        & \\
      &  \\
      &  \\
      % Ubah kalimat-kalimat berikut dengan nama dan NIP dosen penguji pertama
      Dr. Galileo Galilei, S.T., M.Sc.  & (Examiner I) \\
      NIP: 15640215 164201 1 001        & \\
      &  \\
      &  \\
      % Ubah kalimat-kalimat berikut dengan nama dan NIP dosen penguji kedua
      Friedrich Nietzsche, S.T., M.Sc.  & (Examiner II) \\
      NIP: 18441015 190008 1 001        & \\
      &  \\
      &  \\
      % Ubah kalimat-kalimat berikut dengan nama dan NIP dosen penguji ketiga
      Alan Turing, ST., MT.             & (Examiner III) \\
      NIP: 19120623 195406 1 001        & \\
    \end{tabularx}
  \endgroup

  \vspace{4ex}

  \begin{center}
    % Ubah text dibawah menjadi tempat dan tanggal
    \textbf{SURABAYA} \\
    \textbf{May, 2077}
  \end{center}
\endgroup

  \newpage

  \newgeometry{top=0cm}

  \begin{spacing}{1.5}
    % Daftar isi
    \renewcommand*\contentsname{DAFTAR ISI}
    \addcontentsline{toc}{chapter}{\contentsname}
    \tableofcontents
    \newpage

    % Daftar gambar
    \renewcommand*\listfigurename{DAFTAR GAMBAR}
    \addcontentsline{toc}{chapter}{\listfigurename}
    \listoffigures
    \newpage

    % Daftar tabel
    \renewcommand*\listtablename{DAFTAR TABEL}
    \addcontentsline{toc}{chapter}{\listtablename}
    \listoftables
    \newpage
  \end{spacing}

  \restoregeometry

  % Nomor halaman isi dimulai dari sini
  \pagenumbering{arabic}

  % Konten pendahuluan
  \section{JUDUL TUGAS AKHIR}
\vspace{1ex}

% Ubah kalimat berikut sesuai dengan judul tugas akhir
Kalkulasi Energi pada Roket Luar Angkasa Berbasis Anti Gravitasi
\vspace{0.5ex}

\section{RUANG LINGKUP}
\vspace{1ex}

% Ubah daftar berikut sesuai dengan ruang lingkup tugas akhir
\begin{enumerate}[nolistsep]
  \item \textit{Relativitas}
  \item \textit{Fisika Quantum}
\end{enumerate}
\vspace{0.5ex}

\section{LATAR BELAKANG}
\vspace{1ex}

% Ubah paragraf-paragraf berikut sesuai dengan latar belakang dari tugas akhir
Pesatnya perkembangan roket yang merupakan \lipsum[1]
\vspace{0.5ex}

\lipsum[2]
\vspace{0.5ex}

\lipsum[3][1-10]
\vspace{0.5ex}

\section{PERUMUSAN MASALAH}
\vspace{1ex}

% Ubah paragraf berikut sesuai dengan perumusan masalah dari tugas akhir
Masalah dari penelitian ini adalah \lipsum[1][1-6]
\vspace{0.5ex}

\section{TUJUAN TUGAS AKHIR}
\vspace{1ex}

% Ubah paragraf berikut sesuai dengan tujuan tugas akhir
Tujuan dari penelitian ini adalah \lipsum[1][1-6]
\vspace{0.5ex}

  % Konten tinjauan pustaka
  \section{TINJAUAN PUSTAKA}

% Ubah konten-konten berikut sesuai dengan isi dari tinjauan pustaka
\subsection{Hasil penelitian/perancangan terdahulu}
\lipsum[3]

\subsection{Teori/Konsep Dasar}

\subsubsection{Hukum Newton}

% Contoh penggunaan referensi dari pustaka
Newton pernah merumuskan \parencite{Newton1687} bahwa \lipsum[8]
% Contoh penggunaan referensi dari persamaan
Kemudian menjadi persamaan seperti pada persamaan \ref{eq:FirstLaw}.

% Contoh pembuatan persamaan
\begin{equation}
  % Label referensi dari persamaan yang dibuat
  \label{eq:FirstLaw}
  % Baris kode persamaan yang dibuat
  \sum \mathbf{F} = 0\; \Leftrightarrow\; \frac{\mathrm{d} \mathbf{v} }{\mathrm{d}t} = 0.
\end{equation}

\lipsum[9]

\subsubsection{Anti Gravitasi}

\lipsum[10]


  % Konten metodologi
  \chapter{METODOLOGI}

% Ubah konten-konten berikut sesuai dengan isi dari metodologi

\section{Metode yang digunakan}

\lipsum[11]

% Contoh input gambar dengan format *.jpg
\begin{figure} [H] \centering
  % Nama dari file gambar yang diinputkan
  \includegraphics[scale=0.45]{gambar/blueprint.jpg}
  % Keterangan gambar yang diinputkan
  \caption{\emph{Blueprint} roket yang akan diuji coba \parencite{SpaceXBlueprint}}
  % Label referensi dari gambar yang diinputkan
  \label{fig:Blueprint}
\end{figure}

% Contoh penggunaan referensi dari gambar yang diinputkan
Pada \emph{blueprint} yang tertera di Gambar \ref{fig:Blueprint}. \lipsum[12]

\section{Bahan dan peralatan yang digunakan}

\lipsum[13]
\lipsum[3]

\section{Urutan pelaksanaan penelitian}

% Ubah tabel berikut sesuai dengan isi dari rencana kerja
\newcommand{\w}{}
\newcommand{\G}{\cellcolor{gray}}
\begin{table}[H]
  \captionof{table}{Tabel timeline}
  \label{tbl:timeline}
  \begin{tabular}{|p{3.5cm}|c|c|c|c|c|c|c|c|c|c|c|c|c|c|c|c|}

    \hline
    \multirow{2}{*}{Kegiatan} & \multicolumn{16}{|c|}{Minggu}                                                                       \\
    \cline{2-17}              &
    1                         & 2                             & 3  & 4  & 5  & 6  & 7  & 8  & 9  & 10 & 11 & 12 & 13 & 14 & 15 & 16 \\
    \hline

    % Gunakan \G untuk mengisi sel dan \w untuk mengosongkan sel
    Pengambilan data          &
    \G                        & \G                            & \G & \G & \w & \w & \w & \w & \w & \w & \w & \w & \w & \w & \w & \w \\
    \hline

    Pengolahan data           &
    \w                        & \w                            & \w & \w & \G & \G & \G & \G & \w & \w & \w & \w & \w & \w & \w & \w \\
    \hline

    Analisa data              &
    \w                        & \w                            & \w & \w & \w & \w & \w & \w & \G & \G & \G & \G & \w & \w & \w & \w \\
    \hline

    Evaluasi penelitian       &
    \w                        & \w                            & \w & \w & \w & \w & \w & \w & \w & \w & \w & \w & \G & \G & \G & \G \\
    \hline
  \end{tabular}
\end{table}

Pada \emph{timeline} yang tertera di Tabel \ref{tbl:timeline} \lipsum[10]


  % Konten lainnya
  \chapter{HASIL YANG DIHARAPKAN}

\section{Hasil yang Diharapkan dari Penelitian}

Dari penelitian yang akan dilakukan, diharapkan \lipsum[15]

\section{Hasil Pendahuluan}

Sampai saat ini, kami telah \lipsum[16]


  % Daftar pustaka
  \section{DAFTAR PUSTAKA}
  \renewcommand\refname{}
  \vspace{2ex}
  \bibliographystyle{apacite}
  \renewcommand{\bibname}{}
  \begingroup
    \def\chapter*#1{}
    \bibliography{pustaka/pustaka.bib}
  \endgroup


\end{document}
