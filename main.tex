% Pengaturan ukuran teks dan bentuk halaman dua sisi
\documentclass[12pt]{book}

% Pengaturan ukuran halaman dan margin
\usepackage[a4paper,top=30mm,left=30mm,right=20mm,bottom=25mm]{geometry}

% Pengaturan ukuran spasi
\usepackage[singlespacing]{setspace}

% Pengaturan caption untuk tabel
\usepackage{caption}

% Judul dokumen
\title{Proposal Tugas Akhir ITS}
\author{Musk, Elon Reeve}

% Pengaturan detail pada file PDF
\usepackage[pdfauthor={\@author},bookmarksnumbered,pdfborder={0 0 0}]{hyperref}


% Pengaturan ukuran indentasi
\setlength{\parindent}{2em}

% Package lainnya
\usepackage{changepage}
\usepackage{etoolbox} % Mengubah fungsi default

% Pengaturan jenis karakter
\usepackage[utf8]{inputenc}

\usepackage[style=ieee, backend=biber]{biblatex}
\usepackage{enumitem} % Pembuatan list
\usepackage{lipsum} % Pembuatan template kalimat
\usepackage{graphicx} % Input gambar
\usepackage{longtable} % Pembuatan tabel
\usepackage[table,xcdraw]{xcolor} % Pewarnaan tabel
\usepackage{eso-pic} % Untuk menggunakan background image di halaman
\usepackage{txfonts} % Font times
\usepackage{changepage} % Pembuatan teks kolom
\usepackage{multicol} % Pembuatan kolom ganda
\usepackage{multirow} % Pembuatan baris ganda
\usepackage{tabularx} % Untuk mengatur kolom, seperti grid pada CSS
\usepackage{wrapfig}
\usepackage{float}

% Pengaturan format daftar isi, daftar gambar, dan daftar tabel
\usepackage[titles]{tocloft}
\setlength{\cftsecindent}{2em}
\setlength{\cftsubsecindent}{2em}
\setlength{\cftbeforechapskip}{1.5ex}
\setlength{\cftbeforesecskip}{1.5ex}
\setlength{\cftbeforetoctitleskip}{0cm}
\setlength{\cftbeforeloftitleskip}{0cm}
\setlength{\cftbeforelottitleskip}{0cm}
\renewcommand{\cfttoctitlefont}{\hfill\Large\bfseries} % command untuk membuat heading bold dan besar
\renewcommand{\cftaftertoctitle}{\hfill}
\renewcommand{\cftloftitlefont}{\hfill\Large\bfseries}
\renewcommand{\cftafterloftitle}{\hfill}
\renewcommand{\cftlottitlefont}{\hfill\Large\bfseries}
\renewcommand{\cftafterlottitle}{\hfill}

% Definisi untuk "Hati ini sengaja dikosongkan"
\patchcmd{\cleardoublepage}{\hbox{}}{
  \thispagestyle{empty}
  \vspace*{\fill}
  \begin{center}\textit{[Halaman ini sengaja dikosongkan]}\end{center}
  \vfill}{}{}

  % Pengaturan penomoran halaman
\usepackage{fancyhdr}
\fancyhf{}
\renewcommand{\headrulewidth}{0pt}
\pagestyle{fancy}
\fancyfoot[C,CO]{\thepage}
\patchcmd{\chapter}{plain}{fancy}{}{}
\patchcmd{\chapter}{empty}{plain}{}{}

% Pengaturan format judul bab
\usepackage{titlesec}
\renewcommand{\thesection}{\thechapter.\arabic{section}}
\titleformat{\chapter}[hang]{\centering\bfseries\large}{BAB\ \arabic{chapter}\ }{0ex}{\vspace{0ex}\centering}
\titleformat*{\section}{\large\bfseries}
\titleformat*{\subsection}{\normalsize\bfseries}
\titlespacing{\chapter}{0ex}{0ex}{4ex}
\titlespacing{\section}{0ex}{1ex}{0ex}
\titlespacing{\subsection}{0ex}{0.5ex}{0ex}
\titlespacing{\subsubsection}{0ex}{0.5ex}{0ex}
\setcounter{secnumdepth}{3} % Untuk memberi penomoran pada \subsubsection

\counterwithin{figure}{chapter}
\counterwithin{table}{chapter}

% Mengganti figure dan table menjadi gambar dan tabel
\renewcommand{\figurename}{Gambar}
\renewcommand{\tablename}{Tabel}

% Tambahkan format tanda hubung yang benar di sini
\hyphenation{
  ro-ket
  me-ngem-bang-kan
  per-hi-tu-ngan
}

% Menambahkan resource daftar pustaka
\addbibresource{pustaka/pustaka.bib}

% Isi keseluruhan dokumen
\begin{document}
  % Nomor halaman pembuka dimulai dari sini
  \pagenumbering{roman}

  % Atur ulang penomoran halaman
  \setcounter{page}{1}

  % Sampul Bahasa Indonesia
  \newcommand\covercontents{sampul/konten-id.tex}
  \input{sampul/sampul-luar.tex}

  % Lembar pengesahan
  \chapter*{LEMBAR PENGESAHAN}

% Menyembunyikan nomor halaman
\thispagestyle{empty}

\begin{center}
  % Ubah kalimat berikut dengan judul tugas akhir
  \textbf{KALKULASI ENERGI PADA ROKET LUAR ANGKASA BERBASIS \emph{ANTI-GRAVITASI}}
\end{center}

\begingroup
% Pemilihan font ukuran small
\small

\begin{center}
  % Ubah kalimat berikut dengan pernyataan untuk lembar pengesahan
  \textbf{PROPOSAL TUGAS AKHIR} \\
  Diajukan untuk memenuhi salah satu syarat memperoleh gelar
  Sarjana Teknik pada
  Program Studi S-1 Teknik Dirgantara \\
  Departemen Teknik Dirgantara \\
  Fakultas Teknik Dirgantara \\
  Institut Teknologi Sepuluh Nopember
\end{center}

\begin{center}
  % Ubah kalimat berikut dengan nama dan NRP mahasiswa
  Oleh: \textbf{Elon Reeve Musk} \\
  NRP. 0123 20 4000 0001
\end{center}

\begin{center}
  Disetujui Oleh:
\end{center}

\vspace{10ex}

\begingroup
% Menghilangkan padding
\setlength{\tabcolsep}{0pt}

\noindent
\begin{tabularx}{\textwidth}{X c}
  % Ubah kalimat-kalimat berikut dengan nama dan NIP dosen pembimbing pertama
  Nikola Tesla, S.T., M.T.      &                 \\
  NIP: 18560710 194301 1 001    & (Pembimbing)    \\
                                &                 \\
                                &                 \\
                                &                 \\
  % Ubah kalimat-kalimat berikut dengan nama dan NIP dosen pembimbing kedua
  Wernher von Braun, S.T., M.T. &                 \\
  NIP: 19230323 197706 1 001    & (Ko-Pembimbing) \\
\end{tabularx}
\endgroup

\vspace{\fill}

\begin{center}
  Mengetahui,\\
  % Ubah kalimat berikut dengan nama departemen
  Kepala Departemen Teknik Dirgantara FTD-ITS\\
  \vspace{10ex}
  % Ubah kalimat berikut dengan jabatan kepala departemen
  \underline{Nikola Tesla, S.T., M.T. }\\
  NIP 18560710 194301 1 001\\
  \vspace{10ex}
  % Ubah text dibawah menjadi tempat dan tanggal
  \textbf{SURABAYA} \\
  \textbf{Mei, 2077}
\end{center}
\endgroup

  \cleardoublepage

  % Abstrak
  \begin{center}
	\large
  \textbf{KALKULASI ENERGI PADA ROKET LUAR ANGKASA BERBASIS \emph{ANTI-GRAVITASI}}
\end{center}
\addcontentsline{toc}{chapter}{ABSTRAK}
% Menyembunyikan nomor halaman
\thispagestyle{empty}

\begin{flushleft}
    \setlength{\tabcolsep}{0pt}
    \bfseries
    \begin{tabular}{ll@{\hspace{6pt}}l}
    Nama Mahasiswa / NRP&:& Elon Reeve Musk / 0123204000001\\
    Departemen&:& Teknik Dirgantara FTD - ITS\\
    Dosen Pembimbing&:& 1. Nikola Tesla, S.T., M.T.\\
    & & 2. Wernher von Braun, S.T., M.T.\\
    \end{tabular}
    \vspace{4ex}
\end{flushleft}
\textbf{Abstrak}

% Isi Abstrak
Suspensi merupakan komponen penting pada kendaraan bermotor karena berperan
penting dalam menjaga kenyamanan dan keamanan saat berkendara. Sebuah ide baru
diperkenalkan yaitu, Series Active Variable Geometry Suspension (SAVGS), dimana sistem
suspensi ini memiliki performa yang lebih baik dari suspensi pasif dan dapat mengatasi
kelemahan dari suspensi aktif. Penelitian terus dilakukan guna meningkatkan performa dari
SAVGS. Pada penelitian ini akan dipelajari pengaruh panjang linkage (single link) terhadap
performa kendaraan khususnya kenyamanan dan stabilitas. Model seperempat kendaraan
digunakan untuk memodelkan dinamika sistem suspensi kendaraan. Pengaruh panjang single
link dianalisis dalam bentuk koefisien kekakuan dan koefisien peredam. Model linier digunakan
untuk merancang state-feedback control system (LQR). Kinerja sistem kendali diuji pada model
nonlinier yang dibuat dengan menggunakan Simscape Multibody. Hasil simulasi menunjukkan
bahwa semakin panjang single link yang digunakan maka kenyamanan dan stabilitas kendaraan
semakin besar. Namun, semakin panjang single link diperlukan input kontrol yang lebih besar.

\vspace{2ex}
\noindent
\textbf{Kata Kunci: \emph{Roket, Anti-gravitasi, Meong}}
  \cleardoublepage

  \chapter*{ABSTRACT}
\begin{center}
  \large
  \textbf{\emph{ANTI-GRAVITY} BASED ENERGY CALCULATION ON OUTER SPACE ROCKETS}
\end{center}
% Menyembunyikan nomor halaman
\thispagestyle{empty}

\begin{flushleft}
  \setlength{\tabcolsep}{0pt}
  \bfseries
  \begin{tabular}{lc@{\hspace{6pt}}l}
  Student Name / NRP&: &Elon Reeve Musk / 0123204000001\\
  Department&: &Aerospace Engineering FTD - ITS\\
  Advisor&: &1. Nikola Tesla, S.T., M.T.\\
  & & 2. Wernher von Braun, S.T., M.T.\\
  \end{tabular}
  \vspace{4ex}
\end{flushleft}
\textbf{Abstract}

% Isi Abstrak
The abstract must consist between two hundred to three hundred words. \lipsum[1]

\vspace{2ex}
\noindent
\textbf{Keywords: \emph{Rocket, Anti-gravity, Meong}}
  \cleardoublepage

  \begin{spacing}{1.5}
    % Daftar isi
    \renewcommand*\contentsname{DAFTAR ISI}
    \addcontentsline{toc}{chapter}{\contentsname}
    \tableofcontents
    \cleardoublepage

    % Daftar gambar
    \renewcommand*\listfigurename{DAFTAR GAMBAR}
    \addcontentsline{toc}{chapter}{\listfigurename}
    \listoffigures
    \cleardoublepage

    % Daftar tabel
    \renewcommand*\listtablename{DAFTAR TABEL}
    \addcontentsline{toc}{chapter}{\listtablename}
    \listoftables
    \cleardoublepage
  \end{spacing}

  % Nomor halaman isi dimulai dari sini
  \pagenumbering{arabic}

  % Konten pendahuluan
  \section{JUDUL TUGAS AKHIR}
\vspace{1ex}

% Ubah kalimat berikut sesuai dengan judul tugas akhir
Kalkulasi Energi pada Roket Luar Angkasa Berbasis Anti Gravitasi
\vspace{0.5ex}

\section{RUANG LINGKUP}
\vspace{1ex}

% Ubah daftar berikut sesuai dengan ruang lingkup tugas akhir
\begin{enumerate}[nolistsep]
  \item \textit{Relativitas}
  \item \textit{Fisika Quantum}
\end{enumerate}
\vspace{0.5ex}

\section{LATAR BELAKANG}
\vspace{1ex}

% Ubah paragraf-paragraf berikut sesuai dengan latar belakang dari tugas akhir
Pesatnya perkembangan roket yang merupakan \lipsum[1]
\vspace{0.5ex}

\lipsum[2]
\vspace{0.5ex}

\lipsum[3][1-10]
\vspace{0.5ex}

\section{PERUMUSAN MASALAH}
\vspace{1ex}

% Ubah paragraf berikut sesuai dengan perumusan masalah dari tugas akhir
Masalah dari penelitian ini adalah \lipsum[1][1-6]
\vspace{0.5ex}

\section{TUJUAN TUGAS AKHIR}
\vspace{1ex}

% Ubah paragraf berikut sesuai dengan tujuan tugas akhir
Tujuan dari penelitian ini adalah \lipsum[1][1-6]
\vspace{0.5ex}
  \cleardoublepage

  % Konten tinjauan pustaka
  \section{TINJAUAN PUSTAKA}

% Ubah konten-konten berikut sesuai dengan isi dari tinjauan pustaka
\subsection{Hasil penelitian/perancangan terdahulu}
\lipsum[3]

\subsection{Teori/Konsep Dasar}

\subsubsection{Hukum Newton}

% Contoh penggunaan referensi dari pustaka
Newton pernah merumuskan \parencite{Newton1687} bahwa \lipsum[8]
% Contoh penggunaan referensi dari persamaan
Kemudian menjadi persamaan seperti pada persamaan \ref{eq:FirstLaw}.

% Contoh pembuatan persamaan
\begin{equation}
  % Label referensi dari persamaan yang dibuat
  \label{eq:FirstLaw}
  % Baris kode persamaan yang dibuat
  \sum \mathbf{F} = 0\; \Leftrightarrow\; \frac{\mathrm{d} \mathbf{v} }{\mathrm{d}t} = 0.
\end{equation}

\lipsum[9]

\subsubsection{Anti Gravitasi}

\lipsum[10]

  \cleardoublepage

  % Konten metodologi
  \chapter{METODOLOGI}

% Ubah konten-konten berikut sesuai dengan isi dari metodologi

\section{Metode yang digunakan}

\lipsum[11]

% Contoh input gambar dengan format *.jpg
\begin{figure} [H] \centering
  % Nama dari file gambar yang diinputkan
  \includegraphics[scale=0.45]{gambar/blueprint.jpg}
  % Keterangan gambar yang diinputkan
  \caption{\emph{Blueprint} roket yang akan diuji coba \parencite{SpaceXBlueprint}}
  % Label referensi dari gambar yang diinputkan
  \label{fig:Blueprint}
\end{figure}

% Contoh penggunaan referensi dari gambar yang diinputkan
Pada \emph{blueprint} yang tertera di Gambar \ref{fig:Blueprint}. \lipsum[12]

\section{Bahan dan peralatan yang digunakan}

\lipsum[13]
\lipsum[3]

\section{Urutan pelaksanaan penelitian}

% Ubah tabel berikut sesuai dengan isi dari rencana kerja
\newcommand{\w}{}
\newcommand{\G}{\cellcolor{gray}}
\begin{table}[H]
  \captionof{table}{Tabel timeline}
  \label{tbl:timeline}
  \begin{tabular}{|p{3.5cm}|c|c|c|c|c|c|c|c|c|c|c|c|c|c|c|c|}

    \hline
    \multirow{2}{*}{Kegiatan} & \multicolumn{16}{|c|}{Minggu}                                                                       \\
    \cline{2-17}              &
    1                         & 2                             & 3  & 4  & 5  & 6  & 7  & 8  & 9  & 10 & 11 & 12 & 13 & 14 & 15 & 16 \\
    \hline

    % Gunakan \G untuk mengisi sel dan \w untuk mengosongkan sel
    Pengambilan data          &
    \G                        & \G                            & \G & \G & \w & \w & \w & \w & \w & \w & \w & \w & \w & \w & \w & \w \\
    \hline

    Pengolahan data           &
    \w                        & \w                            & \w & \w & \G & \G & \G & \G & \w & \w & \w & \w & \w & \w & \w & \w \\
    \hline

    Analisa data              &
    \w                        & \w                            & \w & \w & \w & \w & \w & \w & \G & \G & \G & \G & \w & \w & \w & \w \\
    \hline

    Evaluasi penelitian       &
    \w                        & \w                            & \w & \w & \w & \w & \w & \w & \w & \w & \w & \w & \G & \G & \G & \G \\
    \hline
  \end{tabular}
\end{table}

Pada \emph{timeline} yang tertera di Tabel \ref{tbl:timeline} \lipsum[10]

  \cleardoublepage

  % Konten lainnya
  \chapter{HASIL YANG DIHARAPKAN}

\section{Hasil yang Diharapkan dari Penelitian}

Dari penelitian yang akan dilakukan, diharapkan \lipsum[15]

\section{Hasil Pendahuluan}

Sampai saat ini, kami telah \lipsum[16]

  \cleardoublepage

  \chapter{JADWAL PENELITIAN}

% Ubah tabel berikut sesuai dengan isi dari rencana kerja
\newcommand{\w}{}
\newcommand{\G}{\cellcolor{gray}}
\begin{table}[H]
  \captionof{table}{Tabel timeline}
  \label{tbl:timeline}
  \begin{tabular}{|p{3.5cm}|c|c|c|c|c|c|c|c|c|c|c|c|c|c|c|c|}

    \hline
    \multirow{2}{*}{Kegiatan} & \multicolumn{16}{|c|}{Minggu}                                                                       \\
    \cline{2-17}              &
    1                         & 2                             & 3  & 4  & 5  & 6  & 7  & 8  & 9  & 10 & 11 & 12 & 13 & 14 & 15 & 16 \\
    \hline

    % Gunakan \G untuk mengisi sel dan \w untuk mengosongkan sel
    Pengambilan data          &
    \G                        & \G                            & \G & \G & \w & \w & \w & \w & \w & \w & \w & \w & \w & \w & \w & \w \\
    \hline

    Pengolahan data           &
    \w                        & \w                            & \w & \w & \G & \G & \G & \G & \w & \w & \w & \w & \w & \w & \w & \w \\
    \hline

    Analisa data              &
    \w                        & \w                            & \w & \w & \w & \w & \w & \w & \G & \G & \G & \G & \w & \w & \w & \w \\
    \hline

    Evaluasi penelitian       &
    \w                        & \w                            & \w & \w & \w & \w & \w & \w & \w & \w & \w & \w & \G & \G & \G & \G \\
    \hline
  \end{tabular}
\end{table}

Pada \emph{timeline} yang tertera di Tabel \ref{tbl:timeline} \lipsum[10]

  \cleardoublepage

  % Daftar pustaka
  \chapter*{DAFTAR PUSTAKA}
  \addcontentsline{toc}{chapter}{DAFTAR PUSTAKA}
  \renewcommand\refname{}
  \vspace{2ex}
  \renewcommand{\bibname}{}
  \begingroup
    \def\chapter*#1{}
    \printbibliography
  \endgroup


\end{document}
